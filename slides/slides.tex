\documentclass[aspectratio=169]{beamer}
\usetheme{Singapore}

% add page numbers at the bottom of the slides
\setbeamertemplate{caption}[numbered]
\addtobeamertemplate{navigation symbols}{}{%
    \usebeamerfont{footline}%
    \usebeamercolor[fg]{footline}%
    \hspace{1em}%
    \raisebox{1.4pt}[0pt][0pt]{\insertframenumber/\inserttotalframenumber}
}

% \definecolor{primarycolor}{HTML}{0000FF}

% \makeatletter

% \def\sectioncolor{primarycolor}% color to be applied to section headers

% \setbeamercolor{palette primary}{use=structure,fg=structure.fg}
% \setbeamercolor{palette secondary}{use=structure,fg=structure.fg!75!black}
% \setbeamercolor{palette tertiary}{use=structure,fg=structure.fg!50!black}
% \setbeamercolor{palette quaternary}{fg=black}

% \setbeamercolor{local structure}{fg=primarycolor}
% \setbeamercolor{structure}{fg=primarycolor}
% \setbeamercolor{title}{fg=primarycolor}
% \setbeamercolor{section in head/foot}{fg=black}

% \setbeamercolor{normal text}{fg=black,bg=white}
% \setbeamercolor{block title alerted}{fg=red}
% \setbeamercolor{block title example}{fg=primarycolor}

% \setbeamercolor{footline}{fg=primarycolor!50}
% \setbeamerfont{footline}{series=\bfseries}

% use classic LaTeX font for maths
\usefonttheme[onlymath]{serif}

\usepackage{cmap}
\usepackage[english]{babel}
\usepackage[T1]{fontenc}
\usepackage[utf8]{inputenc}
\usepackage[kerning=true]{microtype}
\usepackage{lmodern}

\usepackage{amsmath}
\usepackage{amsfonts}
\usepackage{amssymb}
\usepackage{amsthm}

\usepackage{mathtools}
\usepackage{tikz}
\usepackage{xcolor}
\usetikzlibrary{positioning}

\usepackage[
    backend=biber,
    style=numeric,
]{biblatex}
\usepackage{graphicx}
\usepackage[justification=centering]{caption}
\usepackage{csquotes}

\graphicspath{{./images/}}

\addbibresource{../report/report.bib}
% \renewcommand*{\bibfont}{\footnotesize}

\AtBeginSection[]
{
  \begin{frame}
    \frametitle{Plan}
    \tableofcontents[currentsection]
  \end{frame}
}


\theoremstyle{definition}
\newtheorem*{exemple}{Example}

\renewcommand{\leq}{\leqslant}
\renewcommand{\geq}{\geqslant}


\title{\textbf{Adversarial Attacks on Images}}

\author{Pierre-Gabriel Berlureau\and Antoine Groudiev\and Matéo Torrents}

\titlegraphic{\includegraphics[height=1.6cm]{./images/logo-ens-psl.png}}

\date{\today}

\begin{document}
\frame{\titlepage}

\begin{frame}{Plan}
   \tableofcontents
\end{frame}

\section{Adversarial attacks: taxonomy and goals}
\subsection{Adversarial goals}
\begin{frame}{Definition}
  \begin{itemize}
    \item \textbf{Adversarial image}: an image that has been slightly modified to fool a vision system into making a mistake
    \item \textbf{Usual method}: adding a small perturbation to the image
    \begin{equation*}
      X_{\text{attack}} = X_{\text{original}} + \underbrace{\delta X}_{\text{perturbation}}
    \end{equation*}
    with $\delta X$ small
  \end{itemize}
\end{frame}
\begin{frame}{Adversarial goals}
Goals of the attack:
\begin{itemize}
  \item \textbf{Confidence reduction}: reduce the confidence of the model in its prediction
  \item \textbf{Misclassification}: make the model predict a different class
  \item \textbf{Source/target misclassification}: make the model predict a specific class
\end{itemize}
\end{frame}
\subsection{Adversarial capabilities}
\begin{frame}{Training v. testing phase approaches}
  
\end{frame}

\begin{frame}{White-box v. black-box approaches}
  
\end{frame}

\subsection{Real-world examples}
\begin{frame}{Real-world examples}
  
\end{frame}

\section{Attacks algorithms}
\subsection{Fast Gradient Sign Method (FGSM)}
\begin{frame}{Fast Gradient Sign Method (FGSM)}{Classical setup}
  
\end{frame}

\begin{frame}{Fast Gradient Sign Method (FGSM)}{Source/target misclassification}
  
\end{frame}

\begin{frame}{Fast Gradient Sign Method (FGSM)}{Iterating}
  
\end{frame}

\subsection{Facial accessories}
\begin{frame}{Facial accessories}
  
\end{frame}

\section{Defense mechanisms}
\subsection{Adversarial training}
\begin{frame}{Adversarial training}
  
\end{frame}

\subsection{\texttt{NULL} labeling}
\begin{frame}{\texttt{NULL} labeling}
  
\end{frame}

\section{Conclusion}
\begin{frame}{Conclusion}

\end{frame}


\begin{frame}[allowframebreaks]{References}
    \nocite{*}
    \printbibliography
\end{frame}

\end{document}